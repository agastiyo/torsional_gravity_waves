\documentclass[12pt]{article}
\usepackage{physics}

\newcommand{\hodge}{{^\star}}

\begin{document}
\begin{center}
  \Large\textbf{Sanity check on Einstein Equation} \\
  \large\text{10/14/2025}
\end{center}

We start with the Einstein equation found by varying the total action w.r.t. the Vielbein: 

\begin{align}
  \frac{M_{Pl}^2}{2}R^{cd} \wedge V^b \epsilon_{abcd} - 2\mathit{nf}(\dd\vartheta \wedge \dd V_a + \dd\vartheta \wedge \omega_{ba} \wedge V^b) \nonumber \\ = - \frac{1}{2}(\partial^b\vartheta \dd\vartheta \wedge \hodge(V_b \wedge V_a) + \partial_a\vartheta \hodge\dd\vartheta)
\end{align}

Set $n=0$ to remove all torsion. At the end, this should yield the standard Schwarzchild metric:

\begin{align}
  \frac{M_{Pl}^2}{2}R^{cd} \wedge V^b \epsilon_{abcd} &= - \frac{1}{2}(\partial^b\vartheta \dd\vartheta \wedge \hodge(V_b \wedge V_a) + \partial_a\vartheta \hodge\dd\vartheta)
\end{align}

This has simplified to a standard Einstein equation. For the vielbeins, we must choose a spherically symmetrical general metric:

\begin{align}
  \dd s^2 = -e^{\nu(t,r)}\dd{t^2} + e^{\lambda(t,r)}\dd{r^2} + r^2\dd{\theta^2} + r^2\sin^2{\theta}\dd{\vartheta^2}.
\end{align}

Since $V^a = V^a_\mu \dd x^\mu$, we can use the Vielbein of the general metric, calculated in previous notes (Vielbein of the General Schwarzchild Metric):
\begin{align}
  V^a_\mu &= \begin{pmatrix}
    e^{\frac{\nu(t,r)}{2}} & 0 & 0 & 0 \\
    0 & e^{\frac{\lambda(t,r)}{2}} & 0 & 0 \\
    0 & 0 & r & 0 \\
    0 & 0 & 0 & r\sin{\theta}
  \end{pmatrix}
\end{align}

The values of $V^a$ are then:
\begin{align*}
  V^0 &= e^{\frac{\nu(t,r)}{2}} \dd{t} \\
  V^1 &= e^{\frac{\lambda(t,r)}{2}} \dd{r} \\
  V^2 &= r \dd{\theta} \\
  V^3 &= r\sin{\theta} \dd{\vartheta}
\end{align*}

Next, we impose the torsion-free condition to calculate the spin connection:
\begin{align}
  \dd{V^a} + \omega^a_b \wedge V^b = 0
\end{align}
\end{document}