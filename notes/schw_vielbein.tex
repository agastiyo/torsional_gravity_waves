\documentclass[12pt]{article}

\usepackage{physics}

\begin{document}

\section*{Vielbein of the Schwarzchild Metric \\ {\small 7/22/2025}}
The Schwarzchild Metric using natural units ($G = 1$ and $c = 1$):
\begin{align*}
  \dd{s}^2 = -\left(1 - \frac{2M}{r}\right)\dd{t}^2 + \left(1 - \frac{2M}{r}\right)^{-1}\dd{r}^2 + r^2\dd{\theta}^2 + r^2\sin^2{\theta}\dd{\phi}^2
\end{align*}

This gives the following metric tensor:
\begin{align*}
  g_{\mu\nu} =
  \begin{pmatrix}
  -\left(1 - \frac{2M}{r}\right) & 0 & 0 & 0 \\
  0 & \left(1 - \frac{2M}{r}\right)^{-1} & 0 & 0 \\
  0 & 0 & r^2 & 0 \\
  0 & 0 & 0 & r^2 \sin^2{\theta}
\end{pmatrix}
\end{align*}
and it's inverse
\begin{align*}
  g^{\mu\nu} =
  \begin{pmatrix}
  -\left(1 - \frac{2M}{r}\right)^{-1} & 0 & 0 & 0 \\
  0 & 1 - \frac{2M}{r} & 0 & 0 \\
  0 & 0 & \frac{1}{r^2} & 0 \\
  0 & 0 & 0 & \frac{1}{r^2 \sin^2{\theta}}
  \end{pmatrix}
\end{align*}
Using this, I can calculate the forms of the Vielbein $e^a_\mu$, $e^\mu_a$, and $e^{a\mu}$.

Sanity check: The Vielbein is simply a set of basis vectors for each point of the manifold that express our metric. It will satisfy the following equation:
\begin{align*}
  g_{\mu\nu} = \eta_{ab}e^a_\mu e^b_\nu \qquad a,b = 0,1,2,3 \quad \mu,\nu = t,r,\theta,\phi
\end{align*}
where $\eta_{ab} = \text{diag}(-1,1,1,1)$ and is the Minkowski metric. Since the metric is diagonal, the equation can be simplified by saying $a=b$. I will also constrain the Vielbein to being a diagonal matrix, so $e^a_\mu = \text{diag}(e^0_t,e^1_r,e^2_\theta,e^3_\phi)$, which allows me to simplify further by saying $\mu = \nu$. This also gives a correspondance between local and Schwarzchild coordinates, where $a = 0,1,2,3$ correspond to $\mu = t,r,\theta,\phi$ respectively in the Vielbein. Anything else is $0$.

So, I am left with the equation:
\begin{align*}
  g_{\mu\mu} = \eta_{aa}(e^a_\mu)^2
\end{align*}

Solving this explicitly:
\begin{align*}
  g_{tt} &= -\left(1 - \frac{2M}{r}\right) = -(e^0_t)^2 \quad
  &&\rightarrow \quad e^0_t = \left(1 - \frac{2M}{r}\right)^\frac{1}{2} \\
  g_{rr} &= \left(1 - \frac{2M}{r}\right)^{-1} = (e^1_r)^2 \quad
  &&\rightarrow \quad e^1_r = \left(1 - \frac{2M}{r}\right)^{-\frac{1}{2}} \\
  g_{\theta\theta} &= r^2 = (e^2_\theta)^2 \quad
  &&\rightarrow \quad e^2_\theta = r \\
  g_{\phi\phi} &= r^2\sin^2{\theta} = (e^3_\phi)^2 \quad
  &&\rightarrow \quad e^3_\phi = r\sin{\theta}
\end{align*}

So, the Vielbein:
\begin{align*}
  e^a_\mu &= 
  \begin{pmatrix}
    \left(1 - \frac{2M}{r}\right)^\frac{1}{2} & 0 & 0 & 0 \\
    0 & \left(1 - \frac{2M}{r}\right)^{-\frac{1}{2}} & 0 & 0 \\
    0 & 0 & r & 0 \\
    0 & 0 & 0 & r\sin{\theta}
  \end{pmatrix} \\ \\
  e^\mu_a = g^{\mu\nu}\eta_{ab}e^b_\nu &=
  \begin{pmatrix}
    \left(1 - \frac{2M}{r}\right)^{-\frac{1}{2}} & 0 & 0 & 0 \\
    0 & \left(1 - \frac{2M}{r}\right)^\frac{1}{2} & 0 & 0 \\
    0 & 0 & \frac{1}{r} & 0 \\
    0 & 0 & 0 & \frac{1}{r\sin{\theta}}
  \end{pmatrix} \\ \\
  e^{a\mu} = g^{\mu\nu}e^a_\nu &=
  \begin{pmatrix}
    -\left(1 - \frac{2M}{r}\right)^{-\frac{1}{2}} & 0 & 0 & 0 \\
    0 & \left(1 - \frac{2M}{r}\right)^\frac{1}{2} & 0 & 0 \\
    0 & 0 & \frac{1}{r} & 0 \\
    0 & 0 & 0 & \frac{1}{r\sin{\theta}}
  \end{pmatrix}
\end{align*}

\end{document}