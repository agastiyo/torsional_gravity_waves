\documentclass[12pt]{article}
\usepackage{physics}

\begin{document}
\begin{center}
  \Large\textbf{Vielbein of the General Schwarzchild Metric} \\
  \large\text{8/7/2025}
\end{center}

The General Schwarzchild Metric:
\begin{align}
  \dd{s^2} = -e^{\nu(t,r)}\dd{t^2} + e^{\lambda(t,r)}\dd{r^2} + r^2\dd{\theta^2} + r^2\sin^2{\theta}\dd{\phi^2}
\end{align}
Note the the trace of the Minkowski metric is $(-,+,+,+)$. This gives the following Metric Tensor and its inverse:
\begin{align}
  g_{\mu\nu} &= 
  \begin{pmatrix}
    -e^{\nu(t,r)} & 0 & 0 & 0 \\
    0 & e^{\lambda(t,r)} & 0 & 0 \\
    0 & 0 & r^2 & 0 \\
    0 & 0 & 0 & r^2\sin^2{\theta} \\
  \end{pmatrix} \\ \nonumber \\
  g^{\mu\nu} &= 
  \begin{pmatrix}
    -e^{-\nu(t,r)} & 0 & 0 & 0 \\
    0 & e^{-\lambda(t,r)} & 0 & 0 \\
    0 & 0 & \frac{1}{r^2} & 0 \\
    0 & 0 & 0 & \frac{1}{r^2\sin^2{\theta}} \\
  \end{pmatrix}
\end{align}

Calculating the Vielbeins:
\begin{align}
  g_{\mu\nu} = \eta_{ab}e^a_\mu e^b_\nu
\end{align}
Since the metric is diagonal, the equation can be simplified by saying $a=b$. I will also constrain the Vielbein to being a diagonal matrix, which allows me to simplify further by saying $\mu = \nu$. This also gives a correspondance between local and Schwarzchild coordinates, where $a = 0,1,2,3$ correspond to $\mu = t,r,\theta,\phi$ respectively in the Vielbein. Anything else is $0$.

So I am left with:
\begin{align}
  g_{\mu\mu} = \eta_{aa}(e^a_\mu)^2
\end{align}
Solving this explicitly:
\begin{align*}
  g_{tt} &= (-1)(e^0_t)^2 = -e^{\nu(t,r)} &&e^0_t = e^{\frac{\nu(t,r)}{2}} \\
  g_{rr} &= (e^1_r)^2 = e^{\lambda(t,r)} &&e^1_r = e^{\frac{\lambda(t,r)}{2}} \\
  g_{\theta\theta} &= (e^2_\theta)^2 = r^2 &&e^2_\theta = r \\
  g_{\phi\phi} &= (e^3_\phi)^2 = r^2\sin^2{\theta} &&e^3_\phi = r\sin{\theta}
\end{align*}
Therefore, the Vielbeins:
\begin{align*}
  e^a_\mu &= \begin{pmatrix}
    e^{\frac{\nu(t,r)}{2}} & 0 & 0 & 0 \\
    0 & e^{\frac{\lambda(t,r)}{2}} & 0 & 0 \\
    0 & 0 & r & 0 \\
    0 & 0 & 0 & r\sin{\theta}
  \end{pmatrix} \\ \\
  e^\mu_a = g^{\mu\nu}\eta_{ab}e^b_\nu &=
  \begin{pmatrix}
    e^{\frac{-\nu(t,r)}{2}} & 0 & 0 & 0 \\
    0 & e^{\frac{-\lambda(t,r)}{2}} & 0 & 0 \\
    0 & 0 & \frac{1}{r} & 0 \\
    0 & 0 & 0 & \frac{1}{r\sin{\theta}}
  \end{pmatrix} \\ \\
  e^{a\mu} = g^{\mu\nu}e^a_\nu &=
  \begin{pmatrix}
    -e^{\frac{-\nu(t,r)}{2}} & 0 & 0 & 0 \\
    0 & e^{\frac{-\lambda(t,r)}{2}} & 0 & 0 \\
    0 & 0 & \frac{1}{r} & 0 \\
    0 & 0 & 0 & \frac{1}{r\sin{\theta}}
  \end{pmatrix}
\end{align*}

\end{document}