\documentclass[12pt]{article}
\usepackage{physics}

\begin{document}

\begin{center}
  \Large\textbf{Torsion EOMs before substitution} \\
  \large\text{10/13/2025}
\end{center}

Below are the Torsional Equations of Motion in the first half of the mathematica notebook. Only terms with both a left hand and right hand side are included. The rest of the Torsion EOMs trivially show $h=0$, $h^{ij} = 0$, $f^{ij} = 0$. Further proof for this is in Adshead Appendix A.

\begin{align}
2nf r e^{\lambda/2} \vartheta_0' &= -2r (f^{33} + \phi) e^{(\lambda + \nu)/2} \nonumber \\
nf \vartheta_0' &= -(f^{33} + \phi) e^{\nu/2} \nonumber \\
nf \vartheta_0' &= -\phi e^{\nu/2} \nonumber \\
-e^{-\nu/2} nf \vartheta_0' &= \phi
\end{align}

\begin{align}
-2nf r \sin\theta e^{\lambda/2} \vartheta_0' &= 2r \sin\theta (f^{22} + \phi) e^{(\lambda + \nu)/2} \nonumber \\
-nf \vartheta_0' &= (f^{22} + \phi) e^{\nu/2} \nonumber \\
-nf \vartheta_0' &= \phi e^{\nu/2} \nonumber \\
-e^{-\nu/2} nf \vartheta_0' &= \phi
\end{align}

\begin{align}
-2nf r e^{\lambda/2} \vartheta_0' &= 2r (f^{33} + \phi) e^{(\lambda + \nu)/2} \nonumber \\
-nf \vartheta_0' &= (f^{33} + \phi) e^{\nu/2} \nonumber \\
-nf \vartheta_0' &= \phi e^{\nu/2} \nonumber \\
-e^{-\nu/2} nf \vartheta_0' &= \phi
\end{align}

\begin{align}
2nf r^2 \sin\theta \vartheta_0' &= -2r^2 \sin\theta (f^{11} + \phi) e^{\nu/2} \nonumber \\
nf \vartheta_0' &= - (f^{11} + \phi) e^{\nu/2} \nonumber \\
- nf \vartheta_0' &= \phi e^{\nu/2} \nonumber \\
-e^{-\nu/2} nf \vartheta_0' &= \phi
\end{align}

\begin{align}
2nf r \sin\theta e^{\lambda/2} \vartheta_0' &= -2r \sin\theta (f^{22} + \phi) e^{(\lambda + \nu)/2} \nonumber \\
nf \vartheta_0' &= -(f^{22} + \phi) e^{\nu/2} \nonumber \\
-nf \vartheta_0' &= \phi e^{\nu/2} \nonumber \\
-e^{-\nu/2} nf \vartheta_0' &= \phi
\end{align}

\begin{align}
-2 n f r^2 \sin\theta \vartheta_0' &= 2 r^2 \sin\theta (f^{11} + \phi) e^{\nu/2} \nonumber \\
-n f \vartheta_0' &= (f^{11} + \phi) e^{\nu/2} \nonumber \\
-n f \vartheta_0' &= \phi e^{\nu/2} \nonumber \\
-e^{-\nu/2}n f \vartheta_0' &= \phi
\end{align}

In all cases, $\phi$ simplifies to the same form:

\begin{equation}
  \phi = -e^{-\nu/2}n f \vartheta_0'
\end{equation}

Using this and $h=0$, $h^{ij} = 0$, and $f^{ij} = 0$, the Torsion EOMs and Friedman EOMs can be recalculated with greater brevity in the notebook.

\end{document}