\documentclass[12pt]{article}
\usepackage{physics}

\newcommand{\hodge}{{^\star}}

\begin{document}

\begin{center}
  \Large\textbf{Variations on the Action} \\
  \large\text{8/4/2025}
\end{center}

I'll start by considering the total action $S_{tot} = S_{EC} + S_m + S_{NY}$ where
\begin{align}
  S_{EC} &= -\frac{M_{Pl}^2}{4}\int\epsilon_{abcd}V^a \wedge V^b \wedge R^{cd} \\
  S_m &= \frac{1}{2}\int\dd\phi\wedge\hodge\dd\phi \\
  S_{NY} &= -\mathit{nf}\int\dd\phi\wedge T^a \wedge V_a
\end{align}
$S_{EC}$ is the Einstein-Cartan gravitational action, $S_m$ is the scalar field action, and $S_{NY}$ is the coupling to a Nieh-Yan Form. I'll also define:
\begin{align*}
  R^{ab} &= \dd\omega^{ab} + \omega^a_c \wedge \omega^{cb} \\
  T^a &= \dd V^a + \omega^a_b \wedge V^b
\end{align*}
The variation of the total action is
\begin{align}
  \var{S_{tot}} = \var{S_{EC}} + \var{S_m} + \var{S_{NY}}
\end{align}

\section{Vielbein}
\subsection{Einstein-Cartan Action}
\begin{align}
  \var_V{S_{EC}} &= -\frac{M_{Pl}^2}{4}\int \epsilon_{abcd}(\var V^a \wedge V^b + V^a \wedge \var V^b) \wedge R^{cd} \nonumber \\
  &= -\frac{M_{Pl}^2}{4}\int 2\epsilon_{abcd}\var V^a \wedge V^b \wedge R^{cd} \nonumber \\ &\text{(using the antisymmetry of the Levi-Civita Symbol)} \nonumber\\
  &= \boxed{\frac{M_{Pl}^2}{2}\int R^{cd} \wedge V^b \epsilon_{abcd} \wedge \var V^a}
\end{align}

\subsection{Scalar Field Action}
\begin{align}
  \var_V{S_m} &= \frac{1}{2} \int \var\dd\phi \wedge \hodge\dd\phi + \dd\phi \wedge \var\hodge\dd\phi \nonumber \\
  &=  \frac{1}{2} \int \dd\phi \wedge \var\hodge\dd\phi \qquad \text{since } \var_V\dd\phi = 0 
\end{align}
Expanding:
\begin{align*}
  \hodge\dd\phi &= \hodge(\partial_\mu\phi \dd x^\mu) = \hodge(\partial_\mu\phi V^\mu_a V^a)\\ 
  &= \frac{1}{3!} V^\mu_a \partial_\mu\phi \epsilon^a_{bcd}V^b \wedge V^c \wedge V^d
\end{align*}
So,
\begin{align}
  \var_V\hodge\dd\phi &= \frac{1}{3!}(\var V^\mu_a)\partial_\mu\phi \epsilon^a_{bcd} V^b \wedge V^c \wedge V^d + \frac{1}{3!} V^\mu_a \partial_\mu\phi \epsilon^a_{bcd}\var(V^b \wedge V^c \wedge V^d)
\end{align}
Using the identity
\begin{align*}
  (\var V^\mu_a)V^j_\mu &= -(\var V^j_\mu)V^\mu_a \\
  (\var V^\mu_a) &= -(\var V^j_\nu)V^\nu_a V_j^\mu
\end{align*}
and plugging that into eq. (7) and using the antisymmetry of the Levi-Civita Symbol in the second term:
\begin{align*}
  &\var_V\hodge\dd\phi \\ &= -\frac{1}{3!}(\var V^j_\nu)V^\nu_a V_j^\mu \partial_\mu\phi \epsilon^a_{bcd}V^b \wedge V^c \wedge V^d + \frac{1}{3!} V^\mu_a \partial_\mu\phi (3)\epsilon^a_{bcd}\var{V^b} \wedge V^c \wedge V^d \\
  &= -\frac{1}{3!}(\var V^j_\nu)V^\nu_a \partial_j\phi \epsilon^a_{bcd}V^b \wedge V^c \wedge V^d + \frac{1}{2} \partial_a\phi \epsilon^a_{bcd}\var{V^b} \wedge V^c \wedge V^d
\end{align*}
Since we are in an orthonormal frame defined by Vielbeins, we can also say that $\partial_a\phi \epsilon^a_{bcd} = \partial^a\phi \epsilon_{abcd}$. Using this in the second term:
\begin{align*}
  \var_V\hodge\dd\phi &= -\frac{1}{3!}(\var V^j_\nu)V^\nu_a \partial_j\phi \epsilon^a_{bcd}V^b \wedge V^c \wedge V^d + \frac{1}{2} \partial^a\phi \epsilon_{abcd}\var{V^b} \wedge V^c \wedge V^d \\
  &= -(\var V^j_\nu)V^\nu_a \partial_j\phi (\frac{1}{3!} \epsilon^a_{bcd}V^b \wedge V^c \wedge V^d) + \partial^a\phi (\frac{1}{2} \epsilon_{abcd} V^c \wedge V^d) \wedge \var{V^b}
\end{align*}
Finally, using the definition of a dual, we can simplify the terms in parenthesis:
\begin{align*}
  \var_V\hodge\dd\phi &= -(\var V^j_\nu)V^\nu_a \partial_j\phi (\hodge V^a) + \partial^a\phi \hodge(V_a \wedge V_b) \wedge \var{V^b} \\
  &= -\partial_j\phi (\var V^j_a) (\hodge V^a) + \partial^a\phi \var{V^b} \wedge \hodge(V_a \wedge V_b) \\
  &= -\partial_j\phi \hodge(\var V^j_a V^a) + \partial^a\phi \var{V^b} \wedge \hodge(V_a \wedge V_b) \\
  &= -\partial_j\phi \hodge(\var V^j) + \partial^a\phi \var{V^b} \wedge \hodge(V_a \wedge V_b) \\
\end{align*}
Relabeling indices, we are left with
\begin{align}
  \var_V\hodge\dd\phi = \partial^a\phi \var{V^b} \wedge \hodge(V_a \wedge V_b) -\partial_b\phi \hodge(\var V^b)
\end{align}

Plugging this back into eq. (6):
\begin{align*}
  \var_V{S_m} &= \frac{1}{2} \int \dd\phi \wedge (\partial^a\phi \var{V^b} \wedge \hodge(V_a \wedge V_b) -\partial_b\phi \hodge(\var V^b)) \\
   &= \frac{1}{2} \int \partial^a\phi \dd\phi \wedge \var{V^b} \wedge \hodge(V_a \wedge V_b) - \partial_b\phi \dd\phi \wedge \hodge(\var V^b) \\
\end{align*}
Using the identity $A \wedge \hodge B = B \wedge \hodge A$ when $A$ and $B$ have the same dimension in the second term:
\begin{align}
  \var_V{S_m} &= \frac{1}{2} \int \partial^a\phi \dd\phi \wedge \hodge(V_a \wedge V_b) \wedge \var{V^b} - \partial_b\phi (\var V^b) \wedge \hodge\dd\phi \nonumber \\
  &= \boxed{\frac{1}{2} \int (\partial^a\phi \dd\phi \wedge \hodge(V_a \wedge V_b) \wedge \var{V^b} + \partial_b\phi \hodge\dd\phi \wedge \var{V^b})}
\end{align}

\subsection{Nieh-Yan Form}
\begin{align*}
  \var_V{S_{NY}} = -\mathit{nf}\int \dd\phi \wedge (\var{T^a} \wedge V_a + T^a \wedge \var{V_a})
\end{align*}
Expanding:
\begin{align}
  &\var_V{S_{NY}} \nonumber \\ &= -\mathit{nf}\int \dd\phi \wedge ((\var{\dd{V^a}} + \omega^a_b \wedge \var{V^b}) \wedge V_a + (\dd V^a + \omega^a_b \wedge V^b) \wedge \var{V_a}) \nonumber \\
  &= -\mathit{nf}\int \dd\phi \wedge (\var{\dd{V^a}} \wedge V_a + \omega^a_b \wedge \var{V^b} \wedge V_a + \dd V^a \wedge \var{V_a} + \omega^a_b \wedge V^b \wedge \var{V_a}) \nonumber \\
  &= -\mathit{nf}\int \dd\phi \wedge (\var{\dd{V^a}} \wedge V_a + \dd V^a \wedge \var{V_a} + \omega^a_b \wedge (\var{V^b} \wedge V_a + V^b \wedge \var{V_a}))
\end{align}
Focusing on the term with the spin connection:
\begin{align}
  &\omega^a_b \wedge (\var{V^b} \wedge V_a + V^b \wedge \var{V_a}) \nonumber \\
  &= \omega^a_b \wedge \var{V^b} \wedge V_a + \omega^a_b \wedge V^b \wedge \var{V_a} \nonumber \\
  &= \omega^a_b \wedge \var{V^b} \wedge \eta_{ac}V^c + \omega^a_b \wedge V^b \wedge \eta_{ac}\var{V^c} \nonumber \\
  &= \eta_{ac}\omega^a_b \wedge \var{V^b} \wedge V^c + \eta_{ac}\omega^a_b \wedge V^b \wedge \var{V^c} \nonumber \\
  &= \omega_{bc} \wedge \var{V^b} \wedge V^c + \omega_{bc} \wedge V^b \wedge \var{V^c}
\end{align}
Focusing on the second term:
\begin{align*}
  &\omega_{bc} \wedge V^b \wedge \var{V^c} \\
  &=\omega_{cb} \wedge V^c \wedge \var{V^b} \qquad \text{(swapped indices)} \\
  &=-\omega_{cb} \wedge \var{V^b} \wedge V^c \\
  &=\omega_{bc} \wedge \var{V^b} \wedge V^c \\ &\text{(using the antisymmetry of the spin connection)}
\end{align*}
Plugging this back into eq. (11):
\begin{align*}
  \omega_{bc} \wedge \var{V^b} \wedge V^c + \omega_{bc} \wedge \var{V^b} \wedge V^c = 2\omega_{bc} \wedge \var{V^b} \wedge V^c
\end{align*}
Plugging this back into eq.(10):
\begin{align}
  \var_V{S_{NY}} &= -\mathit{nf}\int \dd\phi \wedge (\var{\dd{V^a}} \wedge V_a + \dd V^a \wedge \var{V_a} + 2\omega_{bc} \wedge \var{V^b} \wedge V^c)
\end{align}

Now, focusing on $\var{\dd{V^a}} \wedge V_a + \dd V^a \wedge \var{V_a}$. We can use the Leibniz Rule:
\begin{align*}
  \dd(\var{V^a} \wedge V_a) &= \dd{\var{V^a}} \wedge V_a + (-1)\var{V^a} \wedge \dd{V_a} \\
  &= \dd{\var{V^a}} \wedge V_a - \var{V^a} \wedge \dd{V_a} \\
  &= \dd{\var{V^a}} \wedge V_a - \dd{V_a} \wedge \var{V^a} \\
  &= \dd{\var{V^a}} \wedge V_a - \dd{V^a} \wedge \var{V_a} \\
  \dd(\var{V^a} \wedge V_a) + \dd{V^a} \wedge \var{V_a} &= \var{\dd{V^a}} \wedge V_a
\end{align*}
Plugging this into eq. (12):
\begin{align*}
  &\var_V{S_{NY}} \\ &= -\mathit{nf}\int \dd\phi \wedge (\dd(\var{V^a} \wedge V_a) + \dd{V^a} \wedge \var{V_a} + \dd V^a \wedge \var{V_a} + 2\omega_{bc} \wedge \var{V^b} \wedge V^c) \\
  &= -\mathit{nf}\int \dd\phi \wedge (\dd(\var{V^a} \wedge V_a) + 2\dd V^a \wedge \var{V_a} + 2\omega_{bc} \wedge \var{V^b} \wedge V^c)
\end{align*}
The first term ends up being a boundary term due to, so we can drop the term. Therefore:
\begin{align}
  \var_V{S_{NY}} &= -2\mathit{nf}\int \dd\phi \wedge (\dd V^a \wedge \var{V_a} + \omega_{bc} \wedge \var{V^b} \wedge V^c) \nonumber \\
  &= -2\mathit{nf} \int \dd\phi \wedge \dd V^a \wedge \var{V_a} + \dd\phi \wedge \omega_{bc} \wedge \var{V^b} \wedge V^c \nonumber \\
  &= -2\mathit{nf} \int \dd\phi \wedge \dd V_a \wedge \var{V^a} - \dd\phi \wedge \omega_{bc} \wedge V^c \wedge \var{V^b} \nonumber \\
  &= -2\mathit{nf} \int \dd\phi \wedge \dd V_a \wedge \var{V^a} - \dd\phi \wedge \omega_{cb} \wedge V^b \wedge \var{V^c} \nonumber \\
  &= \boxed{-2\mathit{nf} \int \dd\phi \wedge \dd V_a \wedge \var{V^a} + \dd\phi \wedge \omega_{bc} \wedge V^b \wedge \var{V^c}}
\end{align}

\section{Spin Connection}
\subsection{Einstein-Cartan Action}
\begin{align}
  \var_\omega{S_{EC}} &= -\frac{M_{Pl}^2}{4} \int \epsilon_{abcd}V^a \wedge V^b \wedge \var{R^{cd}} \nonumber \\
  &= -\frac{M_{Pl}^2}{4} \int \epsilon_{abcd}V^a \wedge V^b \wedge (\dd{\var{\omega^{cd}}} + \var{\omega^c_f} \wedge \omega^{fd} + \omega^c_f \wedge \var{\omega^{fd}}) \nonumber \\
  &= -\frac{M_{Pl}^2}{4} \int V^a \wedge V^b \wedge (\epsilon_{abcd}\dd{\var{\omega^{cd}}} + \epsilon_{abcd}\var{\omega^c_f} \wedge \omega^{fd} + \epsilon_{abcd}\omega^c_f \wedge \var{\omega^{fd}})
\end{align}
Simplifying:
\begin{align*}
  \epsilon_{abcd}&\var{\omega^c_f} \wedge \omega^{fd} + \epsilon_{abcd}\omega^c_f \wedge \var{\omega^{fd}} \\
  &= \epsilon_{abcd}\var{\omega^c_f} \wedge \eta^{fi}\omega^d_i + \epsilon_{abcd}\omega^c_f \wedge \eta^{fi}\var{\omega^d_i} \\
  &= \epsilon_{abcd}\eta^{fi} \var{\omega^c_f} \wedge \omega^d_i - \epsilon_{abcd}\eta^{fi} \var{\omega^d_i} \wedge \omega^c_f \\
  &\text{(swapping indices $c \leftrightarrow d$ and $f \leftrightarrow i$)} \\
  &= \epsilon_{abcd}\eta^{fi} \var{\omega^c_f} \wedge \omega^d_i - \epsilon_{abdc}\eta^{if} \var{\omega^c_f} \wedge \omega^d_i \\
  &= \epsilon_{abcd}\eta^{fi} \var{\omega^c_f} \wedge \omega^d_i + \epsilon_{abcd}\eta^{fi} \var{\omega^c_f} \wedge \omega^d_i \\
  &= 2\epsilon_{abcd}\eta^{fi} \var{\omega^c_f} \wedge \omega^d_i \\
  &= 2\epsilon_{abcd} \var{\omega^c_f} \wedge \omega^{fd}
\end{align*}
Plugging this into eq. (14):
\begin{align*}
  \var_\omega{S_{EC}} &= -\frac{M_{Pl}^2}{4} \int V^a \wedge V^b \wedge (\epsilon_{abcd}\dd{\var{\omega^{cd}}} + 2\epsilon_{abcd} \var{\omega^c_f} \wedge \omega^{fd})
\end{align*}
We can drop the total derivative:
\begin{align}
  \var_\omega{S_{EC}} &= -\frac{M_{Pl}^2}{2} \int \epsilon_{abcd} V^a \wedge V^b \wedge \var{\omega^c_f} \wedge \omega^{fd} \nonumber \\
  &= \boxed{\frac{M_{Pl}^2}{2} \int \epsilon_{abcd} V^a \wedge V^b \wedge \omega^{fd} \wedge \var{\omega^c_f}}
\end{align}

\subsection{Scalar Field Action}
$\var_\omega{S_m}$ is trivially $0$.

\subsection{Nieh-Yan Form}
\begin{align*}
  \var_\omega{S_{NY}} &= -\mathit{nf} \int \dd{\phi} \wedge \var{T^a} \wedge V_a \\
  &= -\mathit{nf} \int \dd{\phi} \wedge (\var{\omega^a_b} \wedge V^b) \wedge V_a \\
  &= -\mathit{nf} \int \dd{\phi} \wedge \var{\omega^a_b} \wedge \eta^{bc}V_c \wedge V_a \\
  &= -\mathit{nf} \int \dd{\phi} \wedge \eta^{bc}\var{\omega^a_b} \wedge V_c \wedge V_a \\
  &= -\mathit{nf} \int \dd{\phi} \wedge \var{\omega^{ac}} \wedge V_c \wedge V_a \\
\end{align*}
Antisymmetrizing:
\begin{align}
  \var_\omega{S_{NY}} &= -\frac{\mathit{nf}}{2} \int \dd{\phi} \wedge \var{\omega^{ac}} \wedge V_c \wedge V_a \nonumber \\
  &= \boxed{\frac{\mathit{nf}}{2} \int \dd{\phi} \wedge V_a \wedge V_c \wedge \var{\omega^{ac}}}
\end{align}

I am antisymmetrizing here but not when varying $S_{EC}$. This is because in eq. (15), the spin connections are being summed over with each other and the Levi-Civita Symbol, both of which are antisymmetric. In eq. (16), however, the spin connection is being summed over with two Vielbeins which are not antisymmetric, hence the $\frac{1}{2}$ factor must be added.

\section{Scalar Field}
\subsection{Einstein-Cartan Action}
$\var_\phi{S_{EC}}$ is trivially 0.

\subsection{Scalar Field Action}
\begin{align}
  \var_\phi{S_m} &= \frac{1}{2} \int (\var{\dd{\phi}} \wedge \hodge{\dd{\phi}} + \dd{\phi} \wedge \var{\hodge{\dd{\phi}}}) \nonumber \\
  &= \frac{1}{2} \int (\dd(\var{\phi}) \wedge \hodge{\dd{\phi}} + \dd{\phi} \wedge \hodge{\dd(\var{\phi})})
\end{align}
We can simplify using the identity $A \wedge \hodge B = B \wedge \hodge A$ when $A$ and $B$ have the same dimension.
\begin{align*}
  \dd(\var{\phi}) \wedge \hodge{\dd{\phi}} + \dd{\phi} \wedge \hodge{\dd(\var{\phi})} &= \dd(\var{\phi}) \wedge \hodge{\dd{\phi}} + \dd(\var{\phi}) \wedge \hodge{\dd{\phi}} \\
  &= 2\dd(\var{\phi}) \wedge \hodge{\dd{\phi}}
\end{align*}
Plugging this into eq. (17):
\begin{align}
  \var_\phi{S_m} &= \int \dd(\var{\phi}) \wedge \hodge{\dd{\phi}}
\end{align}
Using the Leibniz Rule:
\begin{align*}
  \dd(\var{\phi} \wedge \hodge{\dd{\phi}}) &= \dd(\var{\phi}) \wedge \hodge{\dd{\phi}} + \var{\phi} \wedge \dd{\hodge{\dd{\phi}}} \\
  &= \dd(\var{\phi}) \wedge \hodge{\dd{\phi}} + \var{\phi}\dd{\hodge{\dd{\phi}}}
\end{align*}
Since $\var{\phi}$ is a 0-form, a wedge product with it is the same as regular multiplication. So:
\begin{align*}
  \dd(\var{\phi}) \wedge \hodge{\dd{\phi}} &= \dd(\var{\phi} \wedge \hodge{\dd{\phi}}) - \var{\phi}\dd{\hodge{\dd{\phi}}}
\end{align*}
Plugging this into eq. (18) and dropping the total derivative:
\begin{align}
  \var_\phi{S_m} &= \boxed{- \int \var{\phi}\dd{\hodge{\dd{\phi}}}}
\end{align}

\subsection{Nieh-Yan Form}
\begin{align}
  \var_\phi{S_{NY}} &= -\mathit{nf} \int \dd(\var{\phi}) \wedge T^a \wedge V_a
\end{align}
Using the Leibniz Rule:
\begin{align*}
  \dd(\var{\phi} \wedge T^a \wedge V_a) &= \dd{\var{\phi}} \wedge T^a \wedge V_a + \var{\phi} \wedge \dd(T^a \wedge V_a) \\
  \dd(\var{\phi} \wedge T^a \wedge V_a) - \var{\phi} \wedge \dd(T^a \wedge V_a) &= \dd{\var{\phi}} \wedge T^a \wedge V_a
\end{align*}
Plugging this into eq. (20) and dropping the total derivative:
\begin{align}
  \var_\phi{S_{NY}} &= -\mathit{nf} \int -\var{\phi} \wedge \dd(T^a \wedge V_a) \nonumber \\
  &= \mathit{nf} \int \var{\phi} \wedge \dd(T^a \wedge V_a) \nonumber \\
  &= \boxed{\mathit{nf} \int \dd(T^a \wedge V_a) \wedge \var{\phi}}
\end{align}

\end{document}