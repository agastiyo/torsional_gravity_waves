\documentclass[12pt]{article}
\usepackage{physics}
\usepackage{amssymb}

\newcommand{\hodge}{{^\star}}

\begin{document}
\begin{center}
  \Large\textbf{Equations of Motion from the Action} \\
  \large\text{8/11/2025}
\end{center}

From my previous notes, I had calculated the Variations on the total action:
\begin{align}
  \var_V{S_{tot}} &= \frac{M_{Pl}^2}{2}\int R^{cd} \wedge V^b \epsilon_{abcd} \wedge \var V^a \nonumber \\ &+ \frac{1}{2} \int (\partial^a\vartheta \dd\vartheta \wedge \hodge(V_a \wedge V_b) \wedge \var{V^b} + \partial_b\vartheta \hodge\dd\vartheta \wedge \var{V^b}) \nonumber \\ &-2\mathit{nf} \int \dd\vartheta \wedge \dd V_a \wedge \var{V^a} + \dd\vartheta \wedge \omega_{bc} \wedge V^b \wedge \var{V^c} \\ \nonumber \\
  \var_\omega{S_{tot}} &= \frac{M_{Pl}^2}{2} \int \epsilon_{abcd} V^a \wedge V^b \wedge \omega^{fd} \wedge \var{\omega^c_f} + \frac{\mathit{nf}}{2} \int \dd{\vartheta} \wedge V_a \wedge V_c \wedge \var{\omega^{ac}} \\ \nonumber \\
  \var_\vartheta{S_{tot}} &= - \int \var{\vartheta}\dd{\hodge{\dd{\vartheta}}} + \mathit{nf} \int \dd(T^a \wedge V_a) \wedge \var{\vartheta}
\end{align}

\section{Vielbein}
\begin{align*}
  \int \frac{M_{Pl}^2}{2}R^{cd} \wedge V^b \epsilon_{abcd} \wedge \var V^a + \frac{1}{2}(\partial^a\vartheta \dd\vartheta \wedge \hodge(V_a \wedge V_b) \wedge \var{V^b} + \partial_b\vartheta \hodge\dd\vartheta \wedge \var{V^b}) \\ - 2\mathit{nf}(\dd\vartheta \wedge \dd V_a \wedge \var{V^a} + \dd\vartheta \wedge \omega_{bc} \wedge V^b \wedge \var{V^c}) = 0
\end{align*}
Relabeling indices:
\begin{align*}
  \int \frac{M_{Pl}^2}{2}R^{cd} \wedge V^b \epsilon_{abcd} \wedge \var V^a + \frac{1}{2}(\partial^b\vartheta \dd\vartheta \wedge \hodge(V_b \wedge V_a) \wedge \var{V^a} + \partial_a\vartheta \hodge\dd\vartheta \wedge \var{V^a}) \\ - 2\mathit{nf}(\dd\vartheta \wedge \dd V_a \wedge \var{V^a} + \dd\vartheta \wedge \omega_{ba} \wedge V^b \wedge \var{V^a}) = 0 \\
  \int \frac{M_{Pl}^2}{2}R^{cd} \wedge V^b \epsilon_{abcd} \wedge \var V^a + \frac{1}{2}(\partial^b\vartheta \dd\vartheta \wedge \hodge(V_b \wedge V_a) + \partial_a\vartheta \hodge\dd\vartheta) \wedge \var{V^a} \\ - 2\mathit{nf}(\dd\vartheta \wedge \dd V_a + \dd\vartheta \wedge \omega_{ba} \wedge V^b) \wedge \var{V^a} = 0 \\
  \int (\frac{M_{Pl}^2}{2}R^{cd} \wedge V^b \epsilon_{abcd} + \frac{1}{2}(\partial^b\vartheta \dd\vartheta \wedge \hodge(V_b \wedge V_a) + \partial_a\vartheta \hodge\dd\vartheta) \\ - 2\mathit{nf}(\dd\vartheta \wedge \dd V_a + \dd\vartheta \wedge \omega_{ba} \wedge V^b)) \wedge \var{V^a} = 0 \\
\end{align*}
Since $\var{V^a}$ is arbitrary:
\begin{align}
  \frac{M_{Pl}^2}{2}R^{cd} \wedge V^b \epsilon_{abcd} + \frac{1}{2}(\partial^b\vartheta \dd\vartheta \wedge \hodge(V_b \wedge V_a) + \partial_a\vartheta \hodge\dd\vartheta) \nonumber \\ - 2\mathit{nf}(\dd\vartheta \wedge \dd V_a + \dd\vartheta \wedge \omega_{ba} \wedge V^b) = 0
\end{align}
so, the Equation of Motion:
\begin{align}
  \frac{M_{Pl}^2}{2}R^{cd} \wedge V^b \epsilon_{abcd} - 2\mathit{nf}(\dd\vartheta \wedge \dd V_a + \dd\vartheta \wedge \omega_{ba} \wedge V^b) \nonumber \\ = - \frac{1}{2}(\partial^b\vartheta \dd\vartheta \wedge \hodge(V_b \wedge V_a) + \partial_a\vartheta \hodge\dd\vartheta)
\end{align}

\section{Spin Connection}
\begin{align*}
  \int \frac{M_{Pl}^2}{2} \epsilon_{abcd} V^a \wedge V^b \wedge \omega^{fd} \wedge \var{\omega^c_f} + \int \frac{\mathit{nf}}{2} \dd{\vartheta} \wedge V_a \wedge V_c \wedge \var{\omega^{ac}} = 0 \\
  \int \frac{M_{Pl}^2}{2} \epsilon_{abcd} V^a \wedge V^b \wedge \omega_f^d \wedge \var{\omega^{fc}} + \int \frac{\mathit{nf}}{2} \dd{\vartheta} \wedge V_a \wedge V_c \wedge \var{\omega^{ac}} = 0
\end{align*}
Relabeling indices:
\begin{align*}
  \int \frac{M_{Pl}^2}{2} \epsilon_{abcd} V^a \wedge V^b \wedge \omega_f^d \wedge \var{\omega^{fc}} + \int \frac{\mathit{nf}}{2} \dd{\vartheta} \wedge V_f \wedge V_c \wedge \var{\omega^{fc}} = 0 \\
  \int (\frac{M_{Pl}^2}{2} \epsilon_{abcd} V^a \wedge V^b \wedge \omega_f^d + \frac{\mathit{nf}}{2} \dd{\vartheta} \wedge V_f \wedge V_c) \wedge \var{\omega^{fc}} = 0 \\
\end{align*}
So,
\begin{align*}
  \frac{M_{Pl}^2}{2} \epsilon_{abcd} V^a \wedge V^b \wedge \omega_f^d + \frac{\mathit{nf}}{2} \dd{\vartheta} \wedge V_f \wedge V_c = 0
\end{align*}

\section{Scalar Field}
\begin{align*}
  - \int \var{\vartheta}\dd{\hodge{\dd{\vartheta}}} + \mathit{nf} \int \dd(T^a \wedge V_a) \wedge \var{\vartheta} = 0
\end{align*}
Simplifying:
\begin{align*}
  \int \var{\vartheta} \left(-\dd{\hodge{\dd{\vartheta}}} + \mathit{nf} \dd(T^a \wedge V_a) \right) = 0
\end{align*}
This implies:
\begin{align*}
  -\dd{\hodge{\dd{\vartheta}}} + \mathit{nf} \dd(T^a \wedge V_a) &= 0 \\
  \mathit{nf} \dd(T^a \wedge V_a) &= \dd{\hodge{\dd{\vartheta}}}
\end{align*}
Writing this in terms of the D'Alembert operator, which can be written as $\square \vartheta = \delta \vartheta = \hodge{\dd}\hodge{\dd\vartheta}$ since $\vartheta$ is a 0-form. Therefore: $\dd \hodge{\dd\vartheta} = \square \vartheta \hodge{1}$. The equation above is written as:
\begin{align*}
  \square \vartheta &= \hodge{\left(\dd(T^a \wedge V_a)\right)} nf \\
  \frac{1}{\sqrt{|g|}} \partial_\mu\left(\sqrt{|g|}g^{\mu\nu}\partial_\nu \vartheta \right) &= nf \frac{1}{4!\sqrt{|g|}}\epsilon^{\mu\nu\rho\sigma}\left(\dd(T^a \wedge V_a)\right)_{\mu\nu\rho\sigma} \\
  \partial_\mu\left(\sqrt{|g|}g^{\mu\nu}\partial_\nu \vartheta \right) &= \frac{nf}{4!}\epsilon^{\mu\nu\rho\sigma}\left(\dd(T^a \wedge V_a)\right)_{\mu\nu\rho\sigma}
\end{align*}

\end{document} 