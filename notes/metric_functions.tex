\documentclass[12pt]{article}
\usepackage{physics}

\begin{document}

\begin{center}
  \Large\textbf{Deriving the Metric Functions} \\
  \large\text{10/18/2025}
\end{center}

\section*{Initial System}

The mathematica notebook provides us with the following equations of motion:
\begin{align}
  e^\alpha(1+r\alpha_r) - e^\beta(e^\alpha+3n^2f^2r^2\vartheta_{0\tau}^2) = \frac{1}{2}r^2e^\beta\vartheta_{0\tau}^2 \\
  e^\alpha(1-r\beta_r) - e^\beta(e^\alpha-3n^2f^2r^2\vartheta_{0\tau}^2) = -\frac{1}{2}r^2e^\beta\vartheta_{0\tau}^2 \\
  e^\alpha\left((2+r\alpha_r)(\alpha_r-\beta_r)+2r\alpha_{rr}\right) - 12e^\beta rn^2f^2\vartheta_{0\tau}^2=2re^\beta \vartheta_{0\tau}^2 \\
  \beta_\tau = \alpha_\tau = \vartheta_{0r} = 0
\end{align}
So, we can say $\alpha = \alpha(r)$, $\beta = \beta(r)$, $\vartheta_0 = \vartheta_0(\tau)$.

Adding (1) and (2) gives:
\begin{align}
  \alpha_r - \beta_r = \frac{2}{r}(e^\beta - 1)
\end{align}
and subtracting (1) and (2) gives:
\begin{align}
  \alpha_r + \beta_r = re^{-\alpha+\beta}\vartheta_{0\tau}^2(1+6n^2f^2)
\end{align}

Assume $u(r) = e^{-\alpha+\beta}$ and $K(\tau) = \vartheta_{0\tau}^2(1+6n^2f^2)$:
\begin{align}
  -\frac{u_r}{u} = \frac{2}{r}(e^\beta-1)
\end{align}
and
\begin{align}
  \alpha_r + \beta_r = ruK
\end{align}
Getting $u_r$ from (9) and $\beta_r$ from (5) and (8), we can see the system is nonlinearly coupled:
\begin{align}
  u_r &= -\frac{2u}{r}(e^\beta-1) \\
  \beta_r &= \frac{1}{2}\left(ruK-\frac{2}{r}(e^{\beta}-1)\right)
\end{align}

Differentiating (10):
\begin{align*}
  \beta_{rr} &= \frac{K}{2}(u+ru_r)-\frac{1}{2}(-\frac{2}{r^2}(e^\beta -1)-\frac{1}{r}(\beta_r e^\beta)) \\
  &= \frac{K(u+ru_r)}{2} + \frac{e^\beta - 1}{r^2} - \frac{\beta_r e^\beta}{r}
\end{align*}
Substituting $u_r$ from (9):
\begin{align*}
  \beta_{rr} &= \frac{Ku(3-2e^\beta)}{2} + \frac{e^\beta - 1}{r^2} - \frac{\beta_r e^\beta}{r}
\end{align*}
From (10), we see $Ku = \frac{2\beta_r}{r} + \frac{2(e^\beta - 1)}{r^2}$:
\begin{align}
  \beta_{rr} &= \frac{\beta_r(3-2e^\beta)}{r} + \frac{(e^\beta-1)(3-2e^\beta)}{r^2} + \frac{e^\beta - 1}{r^2} - \frac{\beta_r e^\beta}{r} \nonumber \\
  &= -\beta_r\frac{3(e^\beta - 1)}{r} + \frac{-2e^{2\beta}+6e^\beta-4}{r^2}
\end{align}

Again, this ODE is unsolvable. Sympy was unable to find a closed-form solution. This only becomes solvable assuming a constant $\beta$, or in a torsionless case.

The python file attempts to numerically solve the system posed in (9) and (10) with assumed initial conditions. Observations are written in comments in the code.

\section*{Solving with Ansatz}

We can define an ansatz $e^\alpha = 1 - \frac{C}{r}$ where $C$ is some constant. $\alpha_r$ is then $\left(1-\frac{C}{r}\right)^{-1}\frac{C}{r^2} = \frac{C}{r(r-C)}$. With this, (5) can be rewritten:
\begin{align*}
  \frac{C}{r(r-C)} - \beta_r &= \frac{2}{r}(e^\beta - 1) \\
  \beta_r &= \frac{C}{r(r-C)} - \frac{2}{r}(e^\beta - 1)
\end{align*}
Let $y(r) = e^\beta$ and $\beta_r = \frac{y'}{y}$:
\begin{align*}
  \frac{y'}{y} &= \frac{C}{r(r-C)} - \frac{2}{r}(y - 1) \\
  y' &= -\frac{2}{r}y^2 + \left(\frac{C}{r(r-C)} + \frac{2}{r}\right)y \\
  y' &= -\frac{2}{r}y^2 + \frac{2r-C}{r(r-C)}y \\
  \frac{y'}{y^2} &= -\frac{2}{r} + \frac{2r-C}{r(r-C)}\frac{1}{y}
\end{align*}
Let $u(r) = \frac{1}{y} = e^{-\beta}$. Then, $u'(r) = -\frac{y'}{y^2}$:
\begin{align*}
  -u' &= -\frac{2}{r} + \frac{2r-C}{r(r-C)}u \\
  u' + \frac{2r-C}{r(r-C)}u &= \frac{2}{r}
\end{align*}

This ODE can be solved using the integrating factor method. First, set our integrating factor:
\begin{align*}
  \mu(r) &= \exp\left(\int \frac{2r-C}{r(r-C)} dr\right) \\
  &= \exp(\ln{r} + \ln|r-C|) \\
  &= r(r-C)
\end{align*}
Using the integrating factor, the ODE can be written as:
\begin{align*}
  (\mu(r)u)' &= \frac{2}{r}\mu(r) \\
  (\mu(r)u)' &= 2(r-C) \\
  \mu(r)u &= \int 2(r-C) dr \\
  ur(r-C) &= 2\left(\frac{1}{2}r^2 - Cr + D\right) \\
  u &= \frac{r^2 - 2Cr + D}{r(r-C)}
\end{align*}
Remembering that $u = e^{-\beta}$:
\begin{align}
  e^\beta = \frac{r(r-C)}{r^2 - 2Cr + D}
\end{align}

This gives us equations for $e^\alpha$ and $e^\beta$:
\begin{align*}
  e^\alpha = 1 - \frac{C}{r} \qquad e^\beta = \frac{r(r-C)}{r^2 - 2Cr + D}
\end{align*}

\section*{Klein-Gordon Equation}

We also have the Klein-Gordon equation:
\begin{align*}
  \partial_\mu\left(\sqrt{-g}g^{\mu\nu}\partial_\nu \vartheta \right) &= \frac{\mathit{nf}}{2} \partial_\mu\left(T_{\nu\rho}^a V_{a\sigma}\right) \varepsilon^{\mu\nu\rho\sigma}
\end{align*}
For this, mathematica gives us:
\begin{align*}
  &e^{\frac{3}{2}\alpha}(4 + r\alpha_r - r\beta_r)(\vartheta_{0r}) + 2e^{\frac{3}{2}\alpha}r\vartheta_{0rr} \\ & \quad + e^{\frac{1}{2}\beta} (e^{\frac{1}{2}(\alpha+\beta)}+6e^{\frac{1}{2}(\alpha+\beta)}n^2f^2)r((\alpha_\tau - \beta_\tau)(\vartheta_{0\tau}) - 2\vartheta_{0\tau\tau}) = 0 \\ \\
  &e^{\frac{1}{2}\beta} (e^{\frac{1}{2}(\alpha+\beta)}+6e^{\frac{1}{2}(\alpha+\beta)}n^2f^2)r(- 2\vartheta_{0\tau\tau}) = 0 \\
  &- e^{\frac{1}{2}\alpha + \beta} (1+6n^2f^2)r(2\vartheta_{0\tau\tau}) = 0
\end{align*}
\begin{align}
  - 2re^{\frac{1}{2}\alpha + \beta} \vartheta_{0\tau\tau}(1+6n^2f^2) = 0
\end{align}

\section*{Torsionless Case}

In the torsionless case, $\vartheta_0 = 0$, so in turn $K(\tau) = 0$. Using (5) and (6):
\begin{align}
  \alpha_r - \beta_r &= \frac{2}{r}(e^\beta - 1) \\
  \alpha_r &= -\beta_r
\end{align}
so,
\begin{align*}
  -2\dv{\beta}{r} &= \frac{2}{r}(e^\beta - 1) \\
  -\int\frac{1}{e^\beta - 1}\dd{\beta} &= \int\frac{1}{r}\dd{r} \\
  -(\ln|e^\beta - 1| - \ln|e^\beta|) &= \ln|r| + C \\
  -\ln|\frac{e^\beta - 1}{e^\beta}| &= \ln|r| + C \\
  \ln|1-e^{-\beta}| &= -\ln|r| + C \\
  1-e^{-\beta} &= Cr^{-1} \\
  e^{-\beta} &= 1-Cr^{-1}
\end{align*}
\begin{align}
  e^\beta &= \left(1-\frac{C}{r}\right)^{-1}
\end{align}

Integrating (15),
\begin{align}
  \alpha &= -\beta + D \nonumber\\
  \alpha &= -\beta \text{ (with a coordinate redefinition)} \nonumber \\
  e^\alpha &= e^{-\beta} \nonumber \\
  e^\alpha &= \left(1-\frac{C}{r}\right)
\end{align}

Plugging this into the General Schwarzchild metric:
\begin{align}
  \dd{s^2} = -\left(1-\frac{C}{r}\right)\dd{t^2} + \left(1-\frac{C}{r}\right)^{-1}\dd{r^2} + r^2\dd{\theta^2} + r^2\sin^2{\theta}\dd{\varphi^2}
\end{align}
Expecting the Newtonian limit at large $r$, we find $C = 2M$ when $G=1$, giving us the Schwarzchild metric.
\end{document}