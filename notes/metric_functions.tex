\documentclass[12pt]{article}
\usepackage{physics}

\begin{document}

\begin{center}
  \Large\textbf{Deriving the Metric Functions} \\
  \large\text{10/18/2025}
\end{center}

\section*{Initial System}

The mathematica notebook provides us with the following equations of motion:
\begin{align}
  e^\alpha(1+r\alpha_r) - e^\beta(e^\alpha+3n^2f^2r^2\vartheta_{0\tau}^2) = \frac{1}{2}r^2e^\beta\vartheta_{0\tau}^2 \\
  e^\alpha(1-r\beta_r) - e^\beta(e^\alpha-3n^2f^2r^2\vartheta_{0\tau}^2) = -\frac{1}{2}r^2e^\beta\vartheta_{0\tau}^2 \\
  e^\alpha\left((2+r\alpha_r)(\alpha_r-\beta_r)+2r\alpha_{rr}\right) - 12e^\beta rn^2f^2\vartheta_{0\tau}^2=2re^\beta \vartheta_{0\tau}^2 \\
  \beta_\tau = \alpha_\tau = \vartheta_{0r} = 0
\end{align}
So, we can say $\alpha = \alpha(r)$, $\beta = \beta(r)$, $\vartheta_0 = \vartheta_0(\tau)$.

Adding (1) and (2) gives:
\begin{align}
  \alpha_r - \beta_r = \frac{2}{r}(e^\beta - 1)
\end{align}
and subtracting (1) and (2) gives:
\begin{align}
  \alpha_r + \beta_r = re^{-\alpha+\beta}\vartheta_{0\tau}^2(1+6n^2f^2)
\end{align}

Assume $u(r) = e^{-\alpha+\beta}$ and $K(\tau) = \vartheta_{0\tau}^2(1+6n^2f^2)$:
\begin{align}
  -\frac{u_r}{u} = \frac{2}{r}(e^\beta-1)
\end{align}
and
\begin{align}
  \alpha_r + \beta_r = ruK
\end{align}
Getting $u_r$ from (9) and $\beta_r$ from (5) and (8), we can see the system is nonlinearly coupled:
\begin{align}
  u_r &= -\frac{2u}{r}(e^\beta-1) \\
  \beta_r &= \frac{1}{2}\left(ruK-\frac{2}{r}(e^{\beta}-1)\right)
\end{align}

\section*{Torsionless Case}

In the torsionless case, $\vartheta_0 = 0$, so in turn $K(\tau) = 0$. Using (5) and (6):
\begin{align}
  \alpha_r - \beta_r &= \frac{2}{r}(e^\beta - 1) \\
  \alpha_r &= -\beta_r
\end{align}
so,
\begin{align*}
  -2\dv{\beta}{r} &= \frac{2}{r}(e^\beta - 1) \\
  -\int\frac{1}{e^\beta - 1}\dd{\beta} &= \int\frac{1}{r}\dd{r} \\
  -(\ln|e^\beta - 1| - \ln|e^\beta|) &= \ln|r| + C \\
  -\ln|\frac{e^\beta - 1}{e^\beta}| &= \ln|r| + C \\
  \ln|1-e^{-\beta}| &= -\ln|r| + C \\
  1-e^{-\beta} &= Cr^{-1} \\
  e^{-\beta} &= 1-Cr^{-1}
\end{align*}
\begin{align}
  e^\beta &= \left(1-\frac{C}{r}\right)^{-1}
\end{align}

Integrating (12),
\begin{align}
  \alpha &= -\beta + D \\
  \alpha &= -\beta \text{ (with a coordinate redefinition)} \\
  e^\alpha &= e^{-\beta} \\
  e^\alpha &= \left(1-\frac{C}{r}\right)
\end{align}

Plugging this into the General Schwarzchild metric:
\begin{align}
  \dd{s^2} = -\left(1-\frac{C}{r}\right)\dd{t^2} + \left(1-\frac{C}{r}\right)^{-1}\dd{r^2} + r^2\dd{\theta^2} + r^2\sin^2{\theta}\dd{\varphi^2}
\end{align}
Expecting the Newtonian limit at large $r$, we find $C = 2M$ when $G=1$, giving us the Schwarzchild metric.
\end{document}