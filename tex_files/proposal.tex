\documentclass[12pt]{article}
\usepackage{physics}
\usepackage{amssymb}

\newcommand{\hodge}{{^\star}}

\begin{document}

\title{Torsion-Induced Corrections to the Schwarzschild Metric from Nieh-Yan Modified Gravity}
\author{Agastya Gaur \and Indranil Das}

\maketitle

\begin{abstract}
  We investigate black hole solutions in Einstein-Cartan gravity with a Nieh-Yan coupling to a scalar field. By evaluating the equations of motion for a Schwarzschild-like ansatz, we derive a torsionful correction to the radial component of the metric. The resulting modification depends on the scalar field's time derivative, introducing a constant rescaling to the radial term while preserving the Schwarzschild form in the absence of torsion.
\end{abstract}

\pagebreak

\section{Actions}

We consider total action $S_{tot} = S_{EC} + S_m + S_{NY}$ composed of the following terms: the Einstein-Cartan gravitational action
\begin{equation}
  S_{EC} = -\frac{M_{Pl}^2}{4}\int\varepsilon_{abcd}V^a \wedge V^b \wedge R^{cd},
\end{equation}
the scalar field action
\begin{equation}
  S_m = \frac{1}{2}\int\dd\vartheta\wedge\hodge\dd\vartheta,
\end{equation}
and the Nieh-Yan coupling
\begin{equation}
  S_{NY} = -\mathit{nf}\int\dd\vartheta\wedge T^a \wedge V_a.
\end{equation}

The curvature and torsion two forms are defined
\begin{align}
  R^{ab} &= \dd\omega^{ab} + \omega^a{}_c \wedge \omega^{cb}, \\
  T^a &= \dd V^a + \omega^a{}_b \wedge V^b.
\end{align}

\subsection{Equations of Motion}

Varying with respect to the Vielbein yields the Einstein equations
\begin{equation}
  \frac{M_{Pl}^2}{2}R^{cd} \wedge V^b \varepsilon_{abcd} - 2\mathit{nf}(\dd\vartheta \wedge \dd V_a + \dd\vartheta \wedge \omega_{ba} \wedge V^b) = \tau_D,
\end{equation}
where
\begin{equation*}
  \tau_D = - \frac{1}{2}(\partial^b\vartheta \dd\vartheta \wedge \hodge(V_b \wedge V_a) + \partial_a\vartheta \hodge\dd\vartheta)
\end{equation*}

Varying with respect to the spin connection yields the torsion constraint
\begin{equation}
  \varepsilon_{abcd} T^c \wedge V^d = -\frac{2nf}{M_{Pl}^2}\dd \vartheta \wedge V_a \wedge V_b
\end{equation}

Varying with respect to the scalar yields the Klein-Gordon equation

\begin{equation}
  \partial_\mu\left(\sqrt{-g}g^{\mu\nu}\partial_\nu \vartheta \right) = \frac{\mathit{nf}}{2} \partial_\mu\left(T_{\nu\rho}{}^a V_{a\sigma}\right) \varepsilon^{\mu\nu\rho\sigma}
\end{equation}

When evaluating the Equations of Motion, we set $M_{Pl} = 1$.

\subsection{Spin Connection}

The spin connection is decomposed into torsion-free and torsion-induced parts

\begin{equation}
  \omega^{ab} = \bar{\omega}^{ab} + \tilde{\omega}^{ab}
\end{equation}
where
\begin{equation*}
  \tilde{\omega}^{0b} = 0 \qquad \tilde{\omega}^{ab} = \phi(\tau)\varepsilon^{ab}{}_k V^k
\end{equation*}

\section{Application to Black Holes}

To model a static, spherically symmetric black hole, we adopt the generalized Schwarzschild metric
\begin{equation}
  \dd{s^2} = -e^{\alpha(\tau,r)}\dd{\tau^2} + e^{\beta(\tau,r)}\dd{r^2} + r^2\dd{\theta^2} + r^2\sin^2{\theta}\dd{\varphi^2}.
\end{equation}
This has the following metric tensor and inverse:
\begin{align}
  g_{\mu\nu} &= 
  \begin{pmatrix}
    -e^{\alpha(\tau,r)} & 0 & 0 & 0 \\
    0 & e^{\beta(\tau,r)} & 0 & 0 \\
    0 & 0 & r^2 & 0 \\
    0 & 0 & 0 & r^2\sin^2{\theta} \\
  \end{pmatrix} \\ \nonumber \\
  g^{\mu\nu} &= 
  \begin{pmatrix}
    -e^{-\alpha(\tau,r)} & 0 & 0 & 0 \\
    0 & e^{-\beta(\tau,r)} & 0 & 0 \\
    0 & 0 & \frac{1}{r^2} & 0 \\
    0 & 0 & 0 & \frac{1}{r^2\sin^2{\theta}} \\
  \end{pmatrix}
\end{align}

The Vielbeins are
\begin{align*}
  V^a{}_\mu &= \begin{pmatrix}
    e^{\frac{\alpha(\tau,r)}{2}} & 0 & 0 & 0 \\
    0 & e^{\frac{\beta(\tau,r)}{2}} & 0 & 0 \\
    0 & 0 & r & 0 \\
    0 & 0 & 0 & r\sin{\theta}
  \end{pmatrix} \\ \\
  V_a{}^\mu = g^{\mu\nu}\eta_{ab}V^b{}_\nu &=
  \begin{pmatrix}
    e^{\frac{-\alpha(\tau,r)}{2}} & 0 & 0 & 0 \\
    0 & e^{\frac{-\beta(\tau,r)}{2}} & 0 & 0 \\
    0 & 0 & \frac{1}{r} & 0 \\
    0 & 0 & 0 & \frac{1}{r\sin{\theta}}
  \end{pmatrix} \\ \\
  V^{a\mu} = g^{\mu\nu}V^a{}_\nu &=
  \begin{pmatrix}
    -e^{\frac{-\alpha(\tau,r)}{2}} & 0 & 0 & 0 \\
    0 & e^{\frac{-\beta(\tau,r)}{2}} & 0 & 0 \\
    0 & 0 & \frac{1}{r} & 0 \\
    0 & 0 & 0 & \frac{1}{r\sin{\theta}}
  \end{pmatrix}
\end{align*}

\subsection{Torsion Constraint}

Evaluating the Torsion constraint with the general Schwarzchild metric and full spin connection we find
\begin{equation}
  \phi(\tau) = -nfe^{-\frac{\alpha}{2}} \vartheta_{\tau}
\end{equation}

\subsection{Einstein Equations}

Evaluating the Einstein equations with the torsion-full spin connection yields the following differential equations:

\begin{align}
  e^\alpha(1+r\alpha_r) - e^\beta(e^\alpha+3n^2f^2r^2\vartheta_{\tau}^2) &= \frac{1}{2}r^2e^\beta\vartheta_{\tau}^2 \\
  e^\alpha(1-r\beta_r) - e^\beta(e^\alpha-3n^2f^2r^2\vartheta_{\tau}^2) &= -\frac{1}{2}r^2e^\beta\vartheta_{\tau}^2 \\
  e^\alpha\left((2+r\alpha_r)(\alpha_r-\beta_r)+2r\alpha_{rr}\right) - 12e^\beta rn^2f^2\vartheta_{\tau}^2&=2re^\beta \vartheta_{\tau}^2 \\
  \beta_\tau = \alpha_\tau = \vartheta_{r} &= 0
\end{align}
So, we can say $\alpha = \alpha(r)$, $\beta = \beta(r)$, $\vartheta_0 = \vartheta_0(\tau)$. We introduce the standard Schwarzschild ansatz for $e^\alpha$:
\begin{equation}
  e^\alpha = a = 1 - \frac{C}{r} \qquad \alpha_r = a^{-1} \frac{C}{r^2}
\end{equation}

Evaluating eq.14 with the ansatz gives us
\begin{align}
  e^\beta = \frac{1}{a} \left( 1 + \frac{r^{2} \vartheta_{\tau}^{2} \left( \tfrac{1}{2} + 3 n^{2} f^{2} \right)}{a} \right)^{-1} \nonumber \\ \nonumber \\
  e^\beta = \left(1 - \frac{C}{r}\right)^{-1} \left( 1 + \frac{r^{2} \vartheta_{\tau}^{2} \left( \tfrac{1}{2} + 3 n^{2} f^{2} \right)}{1 - \frac{C}{r}} \right)^{-1} \nonumber \\
\end{align}
This corresponds to the Schwarzschild radial term modified by a torsion-dependent correction.

\subsection{Klein Gordon Equation}

Evaluating the Klein Gordon Equation yields
\begin{align}
  - 2re^{\frac{1}{2}\alpha + \beta} \vartheta_{\tau\tau}(1+6n^2f^2) = 0
\end{align}
which requires the scalar field's time derivative $\vartheta_\tau$ to be constant.

\section{Torsionful Metric}

The torsionful metric for a black hole then takes the form
\begin{equation}
  \dd{s^2} = -\left(1 - \frac{C}{r}\right)\dd{\tau^2} + \left(1 - \frac{C}{r}\right)^{-1} \mathcal{T} \dd{r^2} + r^2\dd{\theta^2} + r^2\sin^2{\theta}\dd{\varphi^2}
\end{equation}
where $\mathcal{T}$ is the torsional correction term
\begin{equation*}
  \mathcal{T} = \left( 1 + \frac{r^{2} \vartheta_{\tau}^{2} \left( \tfrac{1}{2} + 3 n^{2} f^{2} \right)}{1 - \frac{C}{r}} \right)^{-1}
\end{equation*}

Because $\vartheta_\tau$ is constant, the torsional correction $\mathcal{T}$ remains constant, effectively rescaling the radial metric component.
\end{document}